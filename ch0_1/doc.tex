\section{Введение}
Рассмотрим уравнение Лоуренса:
\begin{equation}
	\begin{array}{ll}
		\dot{x} &= -\sigma x + \sigma y\\
		\dot{y} &= rx - y - xz\\
		\dot{z} &= -bz + xy
	\end{array}
\end{equation}
Здесь $\sigma > 0$ --- число Прандтля, $r > 0$ --- число Релея, $b > 0$ --- геометрический параметр.
Эта система --- странный аттрактор или динамический хаос (Канторово множество в сечениях).
В большинстве случаев 3-х мерные динамические системы ведут себя довольно сложно (много сложнее 2 мерных). И при этом некоторые дискретные системы ведуть себя сложнее непрерывных.
Полезная литература по дискретным системам и биологическим моделям:
\begin{itemize}
	\item И.Г. Петровский, М.В. Федорюк "Дискретные уравнения" физ.-тех. курс
	\item А.С. Братусь, А.С. Новожилов... "Динамические системы и модели биологии"
\end{itemize}
В последней книге подробно разобраны свойства упоминаемых в курсе биологических моделей и их применение.
\subsection{Одномеррные непрерывные системы}
$\dot{N} = \frac{dN}{dt}$ = "рождаемость" - "смертность" - "миграция". 
Одна из первых моделей роста численности населения была описана Мальтусом в 1800:
\begin{equation*}
	\begin{array}{ll}
		\frac{dN}{dt} = aN - bN = rN\\
		N(0) = N_0
	\end{array}
\end{equation*}
В модели Мальтуса население без ограничения на максимальную численность и на еду будет расти экспоненциально.
Также Мальтус предполагал, что продовльствие растет в арифметической прогрессии, а несостоятельность своей модели объяснял различными пороками людей.
На самом деле население растет как $r \frac{dN}{dt} = \frac{N^2}{k^2}$, $r=42$ --- ,$k = 67 \cdot 10^3$ --- размер группы, на которой начинают проявляться коллективные свойства человека. Тогда получим $N(t) = \frac{2 \cdot 10^11}{2026 - t}$ ~ статистически верно. В этой модели ожидается стабилизация населения на 12 миллиардах.
В 1838 Ферклюст предложил иную модель: $\dot{N} = NF(N)$, где $F(N) = r(1-\frac{N}{k})$. Система Ферклюста задает логистическое уравнение. 
Большая часть курса будет посвященна изучению поведения динамических систем и их неподвижных точек. Неподвижная,стационарная точка $\dot{u} = f(u)$ --- это такая $u: f(u) = 0$. Неподвижные точки могут быть атракторами -- центрами притяжения или репеллерами --- одномерными седлами.